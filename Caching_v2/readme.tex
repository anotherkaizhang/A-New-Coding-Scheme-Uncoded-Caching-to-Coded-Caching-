\documentclass[11pt,letterpaper]{article}
\usepackage[margin=1in]{geometry}
\usepackage{ifpdf}
\ifpdf
    \usepackage[pdftex]{graphicx}
    \graphicspath{{figs/}{matlab/}}   % still needed, even with Inkscape-Pdf-Latex
    \usepackage[update]{epstopdf}
\else
	\usepackage{graphicx}
	\graphicspath{{figs/}{matlab/}}   % still needed, even with Inkscape-Pdf-Latex
\fi
%\usepackage{ifpdf}
%\ifpdf
%	\usepackage[pdftex]{graphicx}
%    \usepackage[update]{epstopdf}
%\else
%	\usepackage{graphicx}
%\fi
\usepackage{pdflscape}
\usepackage[]{algorithm2e}
%\usepackage{wrapfig,bbm}
\usepackage{enumitem,color}
\usepackage{amsthm}
\newtheorem{theorem}{Theorem}
\newtheorem{remark}{Remark}
\newtheorem{lemma}{Lemma}
\newtheorem{corollary}{Corollary}
\newtheorem{example}{Example}
\usepackage{hyperref}
\usepackage{array}
\usepackage{amsmath,amsthm}
%\usepackage{amsfonts}
\usepackage{amssymb}
%\usepackage{amstext}
%\usepackage{latexsym}
%\usepackage{color}
\usepackage{cite,setspace}
%\usepackage{multirow}
%\usepackage{breqn}
%%\usepackage{pdflscape}
%%\usepackage{afterpage}
%%\usepackage{capt-of}
%\allowdisplaybreaks
%%\usepackage{flushend}
\usepackage{multirow}
%
%\newtheorem{theorem}{\textbf{Theorem}}
\newtheorem{prop}{\textbf{Proposition}}
%\newtheorem{prop}{\textbf{Proposition}}
%\newtheorem{defn}[theorem]{Definition} 
\newtheorem{defn}{Definition} 
\newtheorem{IEEEproof}{Proof} 
\newtheorem{definition}{\textbf{Definition}}%[section]
%\newtheorem{fact}[theorem]{Fact}
%\newtheorem{property}{\textbf{Property}}%[section]
%\newtheorem{corollary}{\textbf{Corollary}}%[section]
%\newtheorem{claim}{\textbf{Claim}}%[section]
%\newtheorem{lemma}{\textbf{Lemma}}

\newcommand{\Expt}{\mbox{${\mathbb E}$} }
\newcommand{\Expth}{\mbox{$\hat{{\mathbb E}}$} }
\renewcommand{\vec}[1]{\mbox{\boldmath$#1$}}
\newcommand{\SNR}{{\sf SNR}}

\begin{document}
Main function:

1. \texttt{TianYumiddleMR\_v2.m}

\textit{input}: \texttt{N} and \texttt{K};

\textit{output}: a figure showing new corner points.

\begin{itemize}
\item In this function, only \texttt{N} and \texttt{K} in the beginning (line 5) can be customized. The function will output a figure showing new $(M,R)$ corner point(s) (if found) between a pair of points from Tian coding scheme and Yu coding scheme.

\item The function is structured into 6 sections, the first 5 sections aim to storing the matrices \texttt{M}, \texttt{R}, \texttt{Aeq}, etc., for later use in the optimizing process which is in section 6. 

\item The optimization works by solving the LP with the objective function $M+\alpha R$. For a fixed $t$, each iteration finds a new corner point among all demands, that is, the best point that all demands can reach, therefore the effort of performing polygon intersection is saved.

\item The file paths in line 153 and line 173 need to be customized and remain consistent. 
\end{itemize}



\end{document}
